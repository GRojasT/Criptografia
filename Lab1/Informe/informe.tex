%%%%%%%%%%%%%%%%%%%%%%%%%%%%%%%%%%%%%%%%%%%%%%%%%%%%%%%%%%%%%%%%%%%%%%%%%%%%%%
%
% Topico     : Key Shedule
% Autor      : Gustavo Rojas Torres
% Santiago de Chile, 16/04/2018
%
%%%%%%%%%%%%%%%%%%%%%%%%%%%%%%%%%%%%%%%%%%%%%%%%%%%%%%%%%%%%%%%%%%%%%%%%%%%%%%
% 
%
\documentclass{report}
%
%
%
\usepackage{epsfig}
%
\usepackage{pdfpages}
%
\usepackage{amssymb}
%
\renewcommand*\thesection{\arabic{section}}
\newcommand \tab{\hspace*{25 pt}}
\newcommand \minitab{\hspace*{15 pt}}
%
\begin{document}
\begin{titlepage}
\begin{center}
\psfig{figure=L-USACH-16.png,height=3cm,,}
\end{center}
\begin{center}
{\bf Departamento de Matem\'atica y Ciencia de la Computaci\'on}
\end{center}
\vspace{3cm}
\begin{center}
%%%%%%%%%%%%%%%%%%%%%%%%%%%%%%%%%%%%%%%%%%%%%%%%%%%%%%%%%%%%%%%%
%
% MODIFICAR. Despues del tag \bf se coloca el titulo del trabajo
%
{\Large \bf Laboratorio 1 \\
~~ \\
Algoritmo Key Shedule}
%
%%%%%%%%%%%%%%%%%%%%%%%%%%%%%%%%%%%%%%%%%%%%%%%%%%%%%%%%%%%%%%%%
%
\end{center}
\begin{center}
%
%%%%%%%%%%%%%%%%%%%%%%%%%%%%%%%%%%%%%%%%%%%%%%%%%%%%%%%%%%%%%%%%
%
% MODIFICAR. Despues del tag \bf se coloca el semestre y año
%
{\large \bf Primer Semestre 2018}
%
%%%%%%%%%%%%%%%%%%%%%%%%%%%%%%%%%%%%%%%%%%%%%%%%%%%%%%%%%%%%%%%%
%
\end{center}
\vspace{5cm}
\begin{tabular}{c l c}
%
%%%%%%%%%%%%%%%%%%%%%%%%%%%%%%%%%%%%%%%%%%%%%%%%%%%%%%%%%%%%%%%%
%
% MODIFICAR. En el primer campo colocar el nombre de la asignatura y su codigo
%            En el segundo campo colocar el nombre del autor
%
Criptograf\'ia 22633 & ~~~~~~~~~~~~~~~~~ & Gustavo Rojas Torres \\
%
%%%%%%%%%%%%%%%%%%%%%%%%%%%%%%%%%%%%%%%%%%%%%%%%%%%%%%%%%%%%%%%
%
% MODIFICAR. En el primer campo colocar el nombre de la carrera
%            En el segundo campo color direccion electronica
%
Licenciatura en Ciencia de la Computaci\'on & ~~ & gustavo.rojas.t@usach.cl 
%
%%%%%%%%%%%%%%%%%%%%%%%%%%%%%%%%%%%%%%%%%%%%%%%%%%%%%%%%%%%%%%%
%
\end{tabular}
\end{titlepage}
%
\section{Introducci\'on}
El objetivo de este trabajo es an\'alizar e implementar el algoritmo Key Shedule generador de claves para un DES: Data Encryption Standard.
%
\section{Algoritmo implementado}
El algoritmo PC-1.
\newline
\newline
{\bf function} PC-1\\
{\bf precondition}: array[] arreglo de bits, arrayn[] arreglo para las nuevas posiciones en PC-1\\
{\bf postcondition}: arreglo PC-1\\
\newline
\newline
~~
\newline
\begin{tabular}{r l}
\\
1 & \bf{begin function} \\
2 & \minitab cont$\leftarrow$0, init$\leftarrow$56, i$\leftarrow$init\\
3 & \minitab for j$\leftarrow$0 to 28 do\\
4 & \minitab \minitab if j=7 or j=15 or j=23\\
5 & \minitab \minitab \minitab array[init]$\leftarrow$array[57]\\
6 & \minitab \minitab \minitab i$\leftarrow$i+init\\
7 & \minitab \minitab end if\\
8 & \minitab \minitab else\\
9 & \minitab \minitab \minitab i$\leftarrow$i-8\\
10 & \minitab \minitab end else\\
11 & \minitab \minitab cont++\\
12 & \minitab end for\\
13 & \minitab i$\leftarrow$62, v$\leftarrow$0\\
14 & \minitab for j$\leftarrow$28 to 56 do\\
15 & \minitab \minitab array[j]$\leftarrow$array[i]\\
16 & \minitab \minitab if j=35 or j=43 or j=50\\
17 & \minitab \minitab \minitab if v=2\\
18 & \minitab \minitab \minitab \minitab i$\leftarrow$i+23\\
19 & \minitab \minitab \minitab end if\\
20 & \minitab \minitab \minitab else\\
21 & \minitab \minitab \minitab \minitab i$\leftarrow$i+(init-2)\\
22 & \minitab \minitab \minitab \minitab v$\leftarrow$v+1\\
23 & \minitab \minitab \minitab end else\\
24 & \minitab \minitab end if\\
25 & \minitab \minitab else\\
26 & \minitab \minitab \minitab i$\leftarrow$i-8\\
27 & \minitab \minitab end else\\
28 & \minitab end for\\
29 & \bf end function
\end{tabular}
~~~
\newline
\newline
\newline
La clave de 64-bits primero se reduce a 56 bits ignorando cada bit multiplo de 8. En la permutaci\'on PC-1 inicial elimina estos bits.
Los bits eliminados no aumentan el espacio clave\\
\newline
\newline
\newline
El algoritmo PC-2\\
\begin{itemize}
\item[1] La clave resultante de 56 bits se divide en dos mitades $C_0$ y $D_0$.\\
\item[2] Las dos mitades de 28 bits se giran en una o dos posiciones de bit dependiendo de la ronda i.\\
\item[3] Las rondas i=1,2,9,16 las dos mitades se giran hacia la izquierda en un bit. El resto de los casos las dos mitades son giradas a la izquierda por dos bits.\\
\end{itemize}
El n\'umero total de posiciones de rotaci\'on es 28.
%
\section{Formulaci\'on del experimento}
Dado una clave de de 64-bits primero se reduce a 56 bits ignorando cada bit multiplo de 8. En la permutaci\'on PC-1 inicial elimina estos bits. Luego esta clave se divide en dos mitades C0 y D0. Las dos mitades de 28 bits de desplazan c\'iclicamente, es decir, se giran en una o dos posiciones de bit dependiendo de la ronda i seg\'un las siguientes reglas:\\
\begin{itemize}
\item[-] Las rondas i=1,2,9,16 las dos mitades son giradas a la izquierda en un bit.
\item[-] Las rondas i$\neq$1,2,9,16 las dos mitades son giradas a la izquierda por dos bits.\\
\end{itemize}
Las rotaciones solo tienen lugar dentro de la mitad izquierda y derecha. El n\'umero total de posiciones de rotaci\'on es 4x1+12x2=28. Esto lleva a la propiedad que $C_0=C_{16}$ y $D_0=D_{16}$.\\
\newline
Para derivar las claves $k_i$ (de 48 bits cada una) de las 16 rondas del DES, las dos mitades se permutan de nuevo bit a bit utilizando la permutaci\'on PC-2. Los 56 bits de entrada de $C_i$ y $D_i$ e ignora 8 bits de $C_i$ y $D_i$.
%
\section{Curvas de desempe\~no de resultados}
\begin{center}
\psfig{figure=desempeño,height=10cm}
\end{center}
%
\section{Conclusiones}
Si bien el Data Encryption Standard (DES) es considerado inseguro para algunos ataques, dado que el conjunto de claves es muy peque\~no, es el m\'as popular algoritmo de cifrado de bloques. Aun hoy en d\'ia se sigue utilizando en algunas aplicaciones.
%
\section{Forma de Compilaci\'on}
En la terminal, ubicarse en el directorio donde se encuentra el codigo.c
\begin{itemize}
\item[1] gcc lab1.c -o lab1
\item[2] ./lab1
\end{itemize}
%
\end{document}
